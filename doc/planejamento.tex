\documentclass{article}

\usepackage[utf8]{inputenc}
\usepackage{mathtools}

\title{Planejamento}
\author{Renê Cardozo}

\begin{document}
\maketitle
A imagem original lida do arquivo sera copiada para uma imagem corrente, a imagem original nunca será alterada, apenas a
imagem corrente será editada.

Temos as opções:
\begin{itemize}
\item r: pinta as regiões.
\item b: pinta as bordas.
\item o: copia a imagem original novamente na imagem corrente.
\item l: set todos os pixels com a cor de BACK_GROUND em cores.h.
\item g: grava a imagem corrente.
\item clique esquerdo com o mouse: pinta a região a que o ponto pertence.
\item clique direito com o mouse: pinta regiões de mesma cor a que o ponto pertence.
\item a,d,s: aumenta, diminui limiar ou segmenta imagem baseado no novo limiar.
\item x: a execução é encerrada.
\end{itemize}


\section{Segmentação} A segmentação subdivide uma imagem em regiões ou objetos que a compõem. O nível de detalhe em que
a subdivisão é realizada depende do problema a ser resolvido. A imagem  deve ser segmentada em regiões conexas em
relação a pixels de borda. Cada região será formada por pixels de borda ou apenas por pixels que não são de borda e são
limitados por bordas ou pela fronteira da imagem. Dessa forma, um componente central desse EP utilizado para segmentar
imagens será a função pixelBorda(img, limiar, col, lin); que recebe a posição [lin][col] de um pixel de uma imagem img e
retorna TRUE se o pixel [lin][col] é de borda em relação ao valor do inteiro limiar e retorna FALSE em caso contrário.


De maneira geral, em deteccão de bordas são identificados os pixels da imagem em que há uma mudança brusca ou
descontínua na luminosidade. No EP, a luminosidade relativa de um pixel[lin][col] é calculada pela função
luminosidadePixel(), que está implementada no esqueleto do EP.

Para determinar se um pixel é de borda utilizaremos o filtro de Sobel. No método será necessário o cálculo de duas
grandezas, os chamados gradientes horizontal gX e o gradiente vertical gY no pixel. A seguir denotaremos lum[i][j] a
luminosidade do pixel. Calculados os grandiente gX e gY no pixel, a função pixelBorda declara o pixel de borda em
relação a um dado limiar se a norma euclidiana do gradiente for maior que o limiar.


As estruturas usadas para representar imagens, pixels, células de listas de pixels e células de listas de regiões estão
todas definidas no arquivo imagens.h. Cada cor possui o tipo Byte com valores de 0 a 255. A estrutura pixel possui um
array com suas três cores e um apontador para a sua região. A estrutura imagem possuirá o campo width, height e uma
matriz de pixels. Para alocar a matriz, primeiro declara-se um array de pointeiros com (int*) rows com o campo 
\end{document}

